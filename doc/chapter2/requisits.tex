\subsection{Requisits}Abans de dur a terme l'aplicaci�, primer haurem de fer un analisi dels requisits tant funcionals com no funcionals de la aplicaci� desitjada.

Aquest analisi ensservir� per poder cobrir tots els punts que anteriorment hem decididit estudiar de la plataforma android.

\subsubsection{Requisits No Funcionals}
Els requisits no funcionals son el  conjunt de caracter�stiques de qualitat que considerem necessaris a l'hora de dur a terme el disseny i la implementaci� de la nostra aplicaci�.

Los requerimientos no funcionales m�s habituales son la estabilidad, la portabilidad y el costo. 

Ejemplos 

A un sistema se le puede pedir que muestre en tiempo real la cantidad de datos de una base: �se es un requerimiento funcional. En cu�nto tiempo deber�a el sistema actualizar su verificaci�n interna de cantidad de datos es, en cambio, un requerimiento no funcional. 

Requerimientos habituales 

- Disponibilidad 

- Certificaci�n 

- Dependencia de otras partes 

- Documentaci�n 

- Eficiencia 

- Ser extensible 

- Aspectos legales y de licencias 

- Mantenimiento 

- Rendimiento 

- Plataforma 

- Precio 

- Calidad 

- Necesidad de recursos 

- Seguridad 

- Compatibilidad 

- Estabilidad 

- Soporte
\subsubsection{Requisits Funcionals}

-	Independecia de la pantalla :  L'aplicaci� ha d'esser capa� de adaptarse als moviments del aparell en el que esta allotjada, canviant la seva interficie grafica i fent els canvis que es considerin oportuns per una millor presentaci� y aprofitament de la pantalla. Tot aix� independentment del funcionametnt de l'aplicaci�. 

-	Reproducci� de fitxers Mp3: L'aplicaci� ha de poder reproduir fitxers de musica MP3, independentment de caracter�stiques de bitrate i freq��ncia.

-	Informaci� MP3s:	L'aplicaci� ha de poder llegir la informaci� de artista, titol, album... dels tags ID3 dels fitxers MP3s

-	Llibreria Musical: L'aplicaci� ha de poder emmagatzemar persistentment la informaci� del sistema d'informaci�, can�ons, artistes, albums, configuracions, playlists, imatges, lletres.

-	Radio : L'aplicaci� ha de permetre a partir d'un nom d'un artista o d'una paraula, escoltar una emissora de radio corresponent a artistes similars al sol�licitat o en el segon cas, que continguin dita paraula. En cas de voler l'usuari pot marcar la can�� per tornar-la a escoltar quan vulgui.

-	Streaming: L'usuari ha de poder conectar-se a qualsevol font de streaming 

-	Estadisitques Lastfm : L'aplicaci� ha d'interactuar amb lastfm pero mostrar live el is "`PlayingNow"' y pasarli estadistiques de les can�ons reprodu�des, tant en radio com reproductor de fitxers, per que despres l'usuari pugui fer servir tota la comunitat musical de Lastfm.

- Alimentaci� d'APIs externes: L'aplicaci� s'alimentar� de diferents serveis webs externs, per aconseguir informaci� de les cann�ons, tals com portades, fotos dels grups o les lletres de les mateixes. Les APIs a consultar ser�n les de LastFM i una propia que gestiona la cerca de lletres de can�ons.




