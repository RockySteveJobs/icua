\subsection{Requisits}
Abans de dur a terme l'aplicaci�, primer haurem de fer un analisi dels requisits tant funcionals com no funcionals de la aplicaci� desitjada.

Aquest analisi ensservir� per poder cobrir tots els punts que anteriorment hem decididit estudiar de la plataforma android.

\subsubsection{Requisits No Funcionals}
Els requisits no funcionals son el  conjunt de caracter�stiques de qualitat que considerem necessaris a l'hora de dur a terme el disseny i la implementaci� de la nostra aplicaci�.


Los requerimientos no funcionales m�s habituales son la estabilidad, la portabilidad y el costo. 

Ejemplos 


A un sistema se le puede pedir que muestre en tiempo real la cantidad de datos de una base: �se es un requerimiento funcional. En cu�nto tiempo deber�a el sistema actualizar su verificaci�n interna de cantidad de datos es, en cambio, un requerimiento no funcional. 

Requerimientos habituales 
- Disponibilidad 

- Certificaci�n 

- Dependencia de otras partes 

- Documentaci�n 

- Eficiencia 

- Ser extensible 

- Aspectos legales y de licencias 

- Mantenimiento 

- Rendimiento 

- Plataforma 

- Precio 

- Calidad 

- Necesidad de recursos 

- Seguridad 

- Compatibilidad 

- Estabilidad 

- Soporte
